%% start of file `template.tex'.
%% Copyright 2006-2013 Xavier Danaux (xdanaux@gmail.com).
%
% This work may be distributed and/or modified under the
% conditions of the LaTeX Project Public License version 1.3c,
% available at http://www.latex-project.org/lppl/.


\documentclass[11pt,a4paper,sans]{moderncv}        % possible options include font size ('10pt', '11pt' and '12pt'), paper size ('a4paper', 'letterpaper', 'a5paper', 'legalpaper', 'executivepaper' and 'landscape') and font family ('sans' and 'roman')

% moderncv themes
\moderncvstyle{casual}                             % style options are 'casual' (default), 'classic', 'oldstyle' and 'banking'
\moderncvcolor{blue}                               % color options 'blue' (default), 'orange', 'green', 'red', 'purple', 'grey' and 'black'
%\renewcommand{\familydefault}{\sfdefault}         % to set the default font; use '\sfdefault' for the default sans serif font, '\rmdefault' for the default roman one, or any tex font name
\nopagenumbers{}                                  % uncomment to suppress automatic page numbering for CVs longer than one page

% character encoding
\usepackage[utf8]{inputenc}                       % if you are not using xelatex ou lualatex, replace by the encoding you are using
%\usepackage{CJKutf8}                              % if you need to use CJK to typeset your resume in Chinese, Japanese or Korean

% adjust the page margins
\usepackage[scale=0.75]{geometry}
%\setlength{\hintscolumnwidth}{3cm}                % if you want to change the width of the column with the dates
%\setlength{\makecvtitlenamewidth}{10cm}           % for the 'classic' style, if you want to force the width allocated to your name and avoid line breaks. be careful though, the length is normally calculated to avoid any overlap with your personal info; use this at your own typographical risks...

% personal data
\name{Eric}{Brandwein}
\title{Software Engineer \\
\null\hfill\small{01-04-1997} \\
\null\hfill\small{Italian Argentine}}                               % optional, remove / comment the line if not wanted
\address{Virrey Loreto 1799}{1426 Buenos Aires}{Argentina}% optional, remove / comment the line if not wanted; the "postcode city" and and "country" arguments can be omitted or provided empty
\phone[mobile]{+54~(911)~6120~4615}                   % optional, remove / comment the line if not wanted
%\phone[fixed]{+2~(345)~678~901}                    % optional, remove / comment the line if not wanted
%\phone[fax]{+3~(456)~789~012}                      % optional, remove / comment the line if not wanted
\email{brandweineric@gmail.com}                               % optional, remove / comment the line if not wanted
\homepage{http://github.com/ericbrandwein}                         % optional, remove / comment the line if not wanted
%\extrainfo{additional information}                 % optional, remove / comment the line if not wanted
\photo[64pt][0pt]{picture}                       % optional, remove / comment the line if not wanted; '64pt' is the height the picture must be resized to, 0.4pt is the thickness of the frame around it (put it to 0pt for no frame) and 'picture' is the name of the picture file
%\quote{Some quote}                                 % optional, remove / comment the line if not wanted

% to show numerical labels in the bibliography (default is to show no labels); only useful if you make citations in your resume
%\makeatletter
%\renewcommand*{\bibliographyitemlabel}{\@biblabel{\arabic{enumiv}}}
%\makeatother
%\renewcommand*{\bibliographyitemlabel}{[\arabic{enumiv}]}% CONSIDER REPLACING THE ABOVE BY THIS

% bibliography with mutiple entries
%\usepackage{multibib}
%\newcites{book,misc}{{Books},{Others}}
%----------------------------------------------------------------------------------
%            content
%----------------------------------------------------------------------------------
\begin{document}
%\begin{CJK*}{UTF8}{gbsn}                          % to typeset your resume in Chinese using CJK
%-----       resume       ---------------------------------------------------------
\makecvtitle

\section{Summary}
\cvitem{}{
I've been programming since 2010 - professionally since 2015 - mainly developing Android apps in Java. I'm always looking for ways to improve things, and for new Open Source projects to create or to contribute to, while waiting for the next algorithm contest at
\textcolor{blue}{\href{https://codeforces.com/profile/berocs}{Codeforces}}. I'm fluent and self-learnt in Java, Python, C++, and Android, and I possess basic knowledge of many other languages and frameworks, including Node.js, C, Haskell, etc. When a new project pops up, I start contributing with ideas early, and learn fast.
}

\section{Experience}
%\subsection{Vocational}
\cventry{February 2019--February 2020}{Assistant Professor of Algorithms and Data Structures 1}{Computer Science Department, Exact and Natural Sciences Faculty, UBA}{Buenos Aires}{Argentina}{
  Mainly teaching first year students how to write their first imperative paradigm programs in C++.
}%
\cventry{2018--present}{Programming Tutor}{}{Buenos Aires}{Argentina}{
  Teaching Java, Android, and Python programming to high school students.
}%
\cventry{2015--2017}{Software Engineer}{Mercadolibre}{Buenos Aires}{Argentina}{  Three-month training course, learning Java, SQL, HTML, CSS, and others. Then, mainly developing the Android native app, but also working with the Groovy/Grails and Node.js frameworks.  During this time, I almost completely re-developed the company's home view for the Android app.{}%
%Detailed achievements:%
%\begin{itemize}%
%\item Achievement 1;
%\item Achievement 2, with sub-achievements:
%  \begin{itemize}%
%  \item Sub-achievement (a);
%  \item Sub-achievement (b), with sub-sub-achievements (don't do this!);
%    \begin{itemize}
%    \item Sub-sub-achievement i;
%    \item Sub-sub-achievement ii;
%    \item Sub-sub-achievement iii;
%    \end{itemize}
%  \item Sub-achievement (c);
%  \end{itemize}
%\item Achievement 3.
%\end{itemize}
}
%\cventry{year--year}{Job title}{Employer}{City}{}{Description line 1\newline{}Description line 2}
%\subsection{Miscellaneous}
%\cventry{year--year}{Job title}{Employer}{City}{}{Description}


\section{Education}
\cventry{2015--present}{Computer Science Bachelor}{University of Buenos Aires}{Buenos Aires}{Argentina}{
  Currently at fifth year, with a current note average of 8.4 out of 10.
}  % arguments 3 to 6 can be left empty
\cventry{2010--2014}{Bachelor with orientation on Information Technology}{ORT High School}{Buenos Aires}{Argentina}{}

%\section{Master thesis}
%\cvitem{title}{\emph{Title}}
%\cvitem{supervisors}{Supervisors}
%\cvitem{description}{Short thesis abstract}

\section{Languages}
\cvitemwithcomment{Spanish}{Native}{Born and currently living in Argentina}
\cvitemwithcomment{English}{Fluent}{CAE, University of Cambridge}
\cvitemwithcomment{Italian}{Good command}{Bilingual Elementary School, and Italian family}

\section{Technologies}
\cvitem{Languages}{Proficient in Java, C++, and Python. Other languages include Smalltalk, HTML, CSS, C, Javascript, SQL, Kotlin, and Haskell.}
\cvitem{Android Frameworks}{Mockito, Robolectric, and others.}
\cvitem{Other Frameworks}{Django, Node.js, JUnit, Pug.js, express.js, Groovy/Grails.}
\cvitem{Others}{Git, Bash, Linux.}

\section{Other certificates}
\cvlistitem{Third place in the 2019 Argentinian Programming tournament (TAP).}
\cvlistitem{A total of 11 medals and recognitions in the National Argentinian Olympiads of Informatics (OIA) and Mathematics (OMA), including national Gold Medal.}
\cvlistitem{Cambridge Certificate of Advanced English (CAE).}
%\cvlistitem{Item 3. This item is particularly long and therefore normally spans over several lines. Did you notice the indentation when the line wraps?}

\section{Other Projects}
\cvitem{After High School:}{Contributed to many open source projects on different frameworks, including Android and Node.js. These can be seen in \textcolor{blue}{\href{http://github.com/ericbrandwein}{my GitHub page}}.
My most recent contributions where made to \textcolor{blue}{\href{https://github.com/Cuis-Smalltalk/Cuis-Smalltalk-Dev}{Cuis}} and \textcolor{blue}{\href{https://github.com/nsanmartin/qed/}{QED}}, a page where university students can upload and solve exercises.}{}{}{}
\cvitem{During High School:}{
    \begin{itemize}
      \item Android app for converting videos to GIFs, and upload them to social networks;
      \item Obstacle course game made in Unity 3D;
      \item Modified tetris-like game in C\# XNA.
    \end{itemize}}


%\cvlistdoubleitem{Item 2}{Item 5\cite{book1}}
%\cvlistdoubleitem{Item 3}{Item 6. Like item 3 in the single column list before, this item is particularly long to wrap over several lines.}

\section{Interests}
\cvitem{}{Skiing, Singing, Dancing, Tabletop and computer games, and contributing to Open Source projects.}
%\cvitem{hobby 2}{Description}
%\cvitem{hobby 3}{Description}

%\section{References}
%\begin{cvcolumns}
%  \cvcolumn{Category 1}{\begin{itemize}\item Person 1\item Person 2\item Person 3\end{itemize}}
%  \cvcolumn{Category 2}{Amongst others:\begin{itemize}\item Person 1, and\item Person 2\end{itemize}(more upon request)}
%  \cvcolumn[0.5]{All the rest \& some more}{\textit{That} person, and \textbf{those} also (all available upon request).}
%\end{cvcolumns}

% Publications from a BibTeX file without multibib
%  for numerical labels: \renewcommand{\bibliographyitemlabel}{\@biblabel{\arabic{enumiv}}}% CONSIDER MERGING WITH PREAMBLE PART
%  to redefine the heading string ("Publications"): \renewcommand{\refname}{Articles}
%\nocite{*}
%\bibliographystyle{plain}
%\bibliography{publications}                        % 'publications' is the name of a BibTeX file

% Publications from a BibTeX file using the multibib package
%\section{Publications}
%\nocitebook{book1,book2}
%\bibliographystylebook{plain}
%\bibliographybook{publications}                   % 'publications' is the name of a BibTeX file
%\nocitemisc{misc1,misc2,misc3}
%\bibliographystylemisc{plain}
%\bibliographymisc{publications}                   % 'publications' is the name of a BibTeX file

\clearpage
%-----       letter       ---------------------------------------------------------
% recipient data
%\recipient{Company Recruitment team}{Company, Inc.\\123 somestreet\\some city}
%\date{January 01, 1984}
%\opening{Dear Sir or Madam,}
%\closing{Yours faithfully,}
%\enclosure[Attached]{curriculum vit\ae{}}          % use an optional argument to use a string other than "Enclosure", or redefine \enclname
%\makelettertitle
%
%Lorem ipsum dolor sit amet, consectetur adipiscing elit. Duis ullamcorper neque sit amet lectus facilisis sed luctus nisl iaculis. Vivamus at neque arcu, sed tempor quam. Curabitur pharetra tincidunt tincidunt. Morbi volutpat feugiat mauris, quis tempor neque vehicula volutpat. Duis tristique justo vel massa fermentum accumsan. Mauris ante elit, feugiat vestibulum tempor eget, eleifend ac ipsum. Donec scelerisque lobortis ipsum eu vestibulum. Pellentesque vel massa at felis accumsan rhoncus.
%
%Suspendisse commodo, massa eu congue tincidunt, elit mauris pellentesque orci, cursus tempor odio nisl euismod augue. Aliquam adipiscing nibh ut odio sodales et pulvinar tortor laoreet. Mauris a accumsan ligula. Class aptent taciti sociosqu ad litora torquent per conubia nostra, per inceptos himenaeos. Suspendisse vulputate sem vehicula ipsum varius nec tempus dui dapibus. Phasellus et est urna, ut auctor erat. Sed tincidunt odio id odio aliquam mattis. Donec sapien nulla, feugiat eget adipiscing sit amet, lacinia ut dolor. Phasellus tincidunt, leo a fringilla consectetur, felis diam aliquam urna, vitae aliquet lectus orci nec velit. Vivamus dapibus varius blandit.
%
%Duis sit amet magna ante, at sodales diam. Aenean consectetur porta risus et sagittis. Ut interdum, enim varius pellentesque tincidunt, magna libero sodales tortor, ut fermentum nunc metus a ante. Vivamus odio leo, tincidunt eu luctus ut, sollicitudin sit amet metus. Nunc sed orci lectus. Ut sodales magna sed velit volutpat sit amet pulvinar diam venenatis.
%
%Albert Einstein discovered that $e=mc^2$ in 1905.
%
%\[ e=\lim_{n \to \infty} \left(1+\frac{1}{n}\right)^n \]
%
%\makeletterclosing

%\clearpage\end{CJK*}                              % if you are typesetting your resume in Chinese using CJK; the \clearpage is required for fancyhdr to work correctly with CJK, though it kills the page numbering by making \lastpage undefined
\end{document}


%% end of file `template.tex'.
